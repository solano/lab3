\documentclass{beamer}
\usepackage[utf8]{inputenc}

\setbeamercovered{transparent}
\usetheme{Berlin}
\title{Eletrostática}
\subtitle{Tipos de eletrização}
\author[maiara]{Felipe de S. Lincoln \and Person 2 \and Person 3}
\institute[ifsc - usp]{Instituto de Física de São Carlos, USP}
\date{2018}
%\logo{\includegraphics[scale=.02]{gato.jpg}}


\AtBeginSection[]{
Nova seção!!!!!!!!1




}
\begin{document}

\begin{frame}{Titulo do slide}
\titlepage
\end{frame}

\begin{frame}{Sumário}
\tableofcontents[pausesections]
\end{frame}




\section{Introdução}
\begin{frame}{primeiro slide}{oi}
s\cite{chaikin2000principles}
oi\cite{haldane1988model}
sasa\cite{walecka1974theory}
\alert{x=1}\pause
\begin{block}{Equação de newton}
$$ab+3=t$$
\end{block}
\begin{block}{}
$$ab+3=t$$
\end{block}
\begin{block}{\ }
$$ab+3=t$$
\end{block}

\end{frame}
\begin{frame}
Vamos resolver:

$$ax+b$$

depois fazçamos:

$$\Delta = b^2-4ac$$


\end{frame}
\section{Metodologia}
\begin{frame}{primeiro slide}{oi}
oi\cite{pilbratt2010herschel}
\alert{x=1}\pause
\begin{block}{Equação de newton}
$$ab+3=t$$
\end{block}
\begin{block}{}
$$ab+3=t$$
\end{block}
\begin{block}{\ }
$$ab+3=t$$
\end{block}
sasa\cite{4}
\end{frame}
\begin{frame}
Vamos resolver:

$$ax+b$$

depois fazçamos:\cite{lovesey1984theory}

$$\Delta = b^2-4ac$$


\end{frame}

\section*{Referencias}
\begin{frame}[allowframebreaks, shrink=20]
\bibliographystyle{plain}
\bibliography{felipe}
\end{frame}


\end{document}